\documentclass{article}
\usepackage{verbatim}
\usepackage{listings}
\usepackage{fancyhdr}
\pagestyle{fancy}
\lstset{basicstyle=\footnotesize\ttfamily,breaklines=true,frame=lines}

\title{Parser and AST}
\date{}

\begin{document}

\maketitle

\section{Overview}

Compiler is to translate a program written in high-level language,
which is closer to our natural language and intuitive understandings,
to computer understandable assembly code. To achieve this, the
given program will undergo:

\begin{itemize}
  \item Preprocess: Many compilation languages, e.g. C/C++,
    will have this phase to manipulate the texts in programs and remove macros.
  \item Parser: The program will be separated into a stream of tokens, and then
    the parser will structure these tokens for further transformations.
  \item Intermediate Representaion (IR): The parsed program will be emitted to some
    intermediate representations. This representation is closer to assembly codes,
    but easier for optimizations.
  \item Optimizations: The compiler will optimize the IR for better performance, and
    make it ready to code generation.
  \item Code Generation: Generate the IR into assembly codes.
\end{itemize}

In this chapter, the concept of \emph{parsing},
and \emph{abstract/concrete syntax tree} will be covered.
Meanwhile, a useful design pattern, visitor pattern,
which will be widely used in the future development
will also be discussed.

\section{Parse the Program}

\textbf{Parse:} verb. Under the context of engineering a compiler,
it indicates make your compiler (to be implemented) understand the
program to be translated.

\subsection{Context-free Grammar}

\subsection{Abstract Syntax Tree}

\subsection{Symbol Table}

\end{document}
