\documentclass{article}
\usepackage{listings}
\lstset{basicstyle=\footnotesize\ttfamily,breaklines=true}

\title{An Introduction to ECC Lang}
\date{}

\begin{document}

\maketitle

\section{Prerequisites}

Before learning compiler, you should have basic ideas on:
\begin{itemize}
  \item Regular expressions
  \item Context-free grammar
  \item Library linking
  \item Assembly codes
\end{itemize}

\section{Overview}

Educating Compiler Construction Language (ECC Lang) is writtten to define
a language better serve the purpose of compiler education.
The syntaxes and design concepts are inspired by hybridizing C and Java.
To keep the compiler implementation simple, we assume:
\begin{itemize}
  \item All the programs written in this language should \textbf{NOT exceed 1MB}.
    Otherwise, the compiler is not guaranteed/required to output a correct result!
  \item Only single program compilation is supported for now.
\end{itemize}

\section{Language Manual}

An ECC program should be composed by the following aspects:

\begin{itemize}
  \item Function definition.
    \begin{itemize}
      \item \texttt{main} function: The program starts with. This function have no arguments, and return an integeter.
      \item For the better purpose of education, we do NOT support interface declration
	\footnote{Interface declration is actually an legacy from the early stage of computer system design. Because of the
	limited disk/memory size, it is highly desirable to compile the whole program by scanning it only once}.
    \end{itemize}
  \item Class definition.
  \item Global veriable declaration.
\end{itemize}

\section{Exercises}

According to the language manual, define the program in a context-free grammar.

\end{document}
