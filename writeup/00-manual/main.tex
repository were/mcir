\documentclass{article}
\usepackage{verbatim}
\usepackage{listings}
\usepackage{fancyhdr}
\pagestyle{fancy}
\lstset{basicstyle=\footnotesize\ttfamily,breaklines=true,frame=lines}

\title{An Introduction to ECC Lang}
\date{}

\begin{document}

\maketitle

\section{Prerequisites \& Post-gain}

Before learning compiler, you should have basic ideas on:
\begin{itemize}
  \item Regular expressions
  \item Context-free grammar
  \item Assembly codes
  \item Writing multi-file C++ project and linking against external libs.
\end{itemize}

\noindent After going through the whole process, you are supposed to learn:
\begin{itemize}
  \item Basic concepts of compilation including parsing, IR, optimization, and code generation.
  \item Knowing the usage of the most compiler infrastructure LLVM.
  \item Significantly improved coding power after writing a compiler!
\end{itemize}

\section{Overview}

Educating Compiler Construction Language (ECC Lang) is writtten to define
a language better serve the purpose of compiler education.
Listing~\ref{code:hw} shows an example of this language.
The syntaxes and design concepts are inspired by hybridizing C and Java.
This keeps the language implementation simple, so that we can focus on
learning the compiler principles.
We assume:
\begin{itemize}
  \item All the programs written in this language should \textbf{NOT exceed 1MB}.
    Otherwise, the compiler is not guaranteed/required to output a correct result!
  \item Only single program compilation is supported for now.
\end{itemize}

\section{Language Manual}

\begin{lstlisting}[caption=Hello world!\label{code:hw}, language=C]
int main() {
  // This function can be invoked before declaration.
  printHello();
}
void printHello() {
  println("Hello world!");
}
\end{lstlisting}

\subsection{Overview}

An ECC program should be composed by the following aspects:

\begin{itemize}
  \item Function definition.
    \begin{itemize}
      \item \texttt{main} function: The program starts with. This function have no arguments, and return an integeter.
      \item For the better purpose of education, we do NOT support interface declration
	\footnote{Interface declration is actually an legacy from the early stage of computer system design. Because of the
	limited disk/memory size, it is highly desirable to compile the whole program by scanning it only once}.
    \end{itemize}
  \item Class definition.
  \item Global veriable declaration.
\end{itemize}

\subsection{Comments}

We only support \texttt{//} to comment a line. No \texttt{/**/} supported.

\subsection{Data Types (\& their Constants)}
Declaring a varialbe is just as simple as \texttt{\{type\} \{id\} [= \{initializer\}];},
and the initializer is optional. The \texttt{id} of a variable should not start with a
number, and it can be composed by a combination of numbers, letters, and underscore.
To keep the syntax simple, we do not support declaring multiple variables separated by
commas (\texttt{,}).

\subsubsection{Builtin Types}
These following types are builtin types:
\begin{itemize}
  \item \texttt{void}: Cannot be an variable, can only be the return type of a function.
  \item \texttt{int}: A 32-bit signed integer. Constant integers can range from $[-2^{31},2^{31}-1]$.
  \item \texttt{char}: A 8-bit char. To keep it simple, all the char surrounded by a pair of
    \texttt{'}s should be printable.
    Only 4 backslash escape characters are supported, $\backslash$\texttt{'}, $\backslash$\texttt{"},
    $\backslash\backslash$, and $\backslash$\texttt{n}.
  \item \texttt{bool}: A boolean value, whose constants can be either \texttt{true},or \texttt{false}.
    Unlike C, there is no implicit conversion to bool for all the expressions (\texttt{int}, \texttt{char}, o \texttt{classes}).
  \item \texttt{string}: Literals surrounded by \texttt{"} are constants of strings. Just like
    \texttt{char}, each char of a string should either be printable or supported escape characters.
    Strings are immutable.
    The string data type with three builtin members:
    \begin{itemize}
      \item \texttt{int size()}: return the length of this string.
      \item \texttt{int parseInt()}: convert the string into an integer.
      \item \texttt{char at(int pos)}: starting with 0, return the character at the given position.
    \end{itemize}
\end{itemize}

Note \texttt{int}, \texttt{char}, and \texttt{bool} are plain old data (POD), so they have instances.
\texttt{string} is non-modifiable, so its behavior of being a POD or a class,
does not matter that much.


\subsubsection{Classes}

We also allow users to define their classes and classes can have their member functions.
Listing~\ref{code:class} shows an example of defining a class.
The class name has the same requirement as the variable id.
Unlike conventional C, a class can only be a pointer to an instance.
This design concept is widely adopted in modern languages, like Java and Python.
Pointers can be \texttt{null} when empty. As mentioned before, there is no implicit conversion
to \texttt{0} for \texttt{null}.

\begin{lstlisting}[caption=Class definition\label{code:class}]
class A {
  int value;
  // Constructor is similar syntax to C,
  // but have no ": ()" initialization.
  A(int v) {
    this.value = v;
  }
  void print() {
    println(this.value);
  }
};

int main() {
  A a = new A(5);
  A b = a;
  a.print(); // prints 5
  b.a = 1;
  a.print(); // prints 1
  return 0;
}
\end{lstlisting}

TODO(@were): support destructor, inherence, virtual function, and interfacing.

\subsubsection{Arrays}

Array allocation is very similar in what we have in Java.
Arrays have one builtin method \texttt{size()} for the size of the array.

\begin{lstlisting}[caption=1-D Array Allocation]
int[] a = new int[128];
\end{lstlisting}

We support two types of jagged array allocation.
\begin{lstlisting}[caption=1-D Array Allocation]
int[][] a = new int[128][128];
int[][] b = new int[128][];
for (int i = 0; i < 128; ++i) {
  b[i] = new int[i + 1];
}
\end{lstlisting}

We also support compound arrays.
\begin{lstlisting}[caption=1-D Array Allocation]
class A {
  // ...
};
// Allocate 128 empty pointers of A.
A[] a = new A[128];
\end{lstlisting}

\subsection{Expressions}

\begin{itemize}
  \item Arithmetic Opertions: \texttt{+, -, *, /}
  \item Bitwise Opertions: \verb|&, |\texttt{|, }\verb|^, ~|
  \item Logic Operations: \verb|&&, |\texttt{||, }\verb|, !|
  \item Increase, Decrease: \texttt{++, --}
  \item Access Attributes: \texttt{.}
  \item Pranthesis: \texttt{()}
  \item Brackets: \texttt{[]}
  \item Negative: \texttt{-}
  \item Comparison: \verb|==, >, <|
  \item Assignment: \texttt{=}
    \begin{itemize}
      \item For simplicity, unlike C, assignment has no return value.
	Therefore, no \texttt{a=b=0} allowed.
    \end{itemize}
\end{itemize}

The priority of these operations are the same as C.

\section{Statements}

Every statement can be a declaration, an assignment,
a conditional statement, for-loop,
while-loop, or a compound statement.

\begin{lstlisting}[caption=Conditional statement]
if (a > b) {
  // do a
} else {
  // do b
}
\end{lstlisting}

\begin{lstlisting}[caption=For loop]
for (int i = 0; i < n; ++i) {
  // do something
}
\end{lstlisting}

\begin{lstlisting}[caption=While loop]
while (cond) {
  // do something
}
\end{lstlisting}


Unlike C, though compound statement does open a new scope,
we cannot define variables with same id to hide instances in outer scopes,
as it is shown in Listing~\ref{code:hide}.
\begin{lstlisting}[caption=An example of scope\label{code:hide}]
int a;
{
  int a; // This is not allowed!
  int b; // b is dedicated to this scope.
}
// We cannot use b here.
\end{lstlisting}

\subsection{Function}

Function declaration is the same C, but we allow to define new sub-functions
within a function that captures all the values in that scope. If the return
type is not void, the compiler should not pass the semantic check.

\begin{lstlisting}[caption=While loop]
{return-type} {func-id}({arg-list}) {
  // do something.
  // you can still define new functions here.
}
\end{lstlisting}

\subsection{Key Words}

To sum up, these key words are reserved \texttt{bool int string null void true false if for while break continue return new class this}.

\section{Exercises}

According to the language manual, define the program in a context-free grammar.

\end{document}
